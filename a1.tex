\documentclass[a4paper,10pt]{article}
%\documentclass[a4paper,10pt,landscape]{article}
\usepackage[top=2.5cm,bottom=2.5cm,left=2.5cm,right=2.5cm,showframe]{geometry}
\usepackage{xcolor,fancyhdr}
\usepackage{tikz}
\usepackage{amsmath,amssymb,amsthm,amsfonts,physics}
\usepackage{stmaryrd}
\include{stylefile}

\usepackage{lineno}
\linenumbers
\setpagewiselinenumbers

%%%%%%%%%%%%%  PLEASE DO NOT EDIT ANY OF THE LINES ABOVE %%%%%%%%%%%%%%%
% Insert your text between "\begin{document}" and "\end{document}" below. 
% The total length of your summary notes should not exceed 2 sides of a
% single sheet of A4, with maximum 58 lines of text per page.
%%%%%%%%%%%%%%%%%%%%%%%%%%%%%%%%%%%%%%%%%%%%%%%%%%%%%%%%%%%%%%%%%%%%%%%%

\definecolor{fg}{RGB}{34,139,34}
\begin{document} \noindent
\textcolor{red}{APDE:} \textbf{Charpit:} $F(p,q,u,x,y)=0$ with $\color{blue}{u_x = p,u_y=q,\dot x= F_p,\dot y=F_q}$. Then via $F_x,F_y, \& \: p_y=q_x \rightarrow \color{blue}{p_\tau = -F_x - pF_u}$, $\color{blue}{q_\tau = -F_y-qF_u}, \color{blue}{u_\tau = pF_p + qF_q}$. Also, $\color{blue}{u0_s = p_0x0_s + q_0y0_s; F_0 = 0}$ - last 2 needed to show $u$ defined on $\Gamma$.
\textbf{Max Principle:} For $- \Delta u = f \leq 0 \rightarrow \max u \in \partial D$. First show contradiction assuming $LU = f < 0$, then try some auxillary function $\psi = U + \alpha\left( T_{\max} \right) g\left( x_i,y_i \right)$ s.t. $L\psi < 0$ so $\max \psi = \max_{\in \partial D} \psi$. Gets $\max e_{i,j}$; change to $-\alpha$ for $\min e_{i,j}$.
\textbf{Laplacian:} In $2D: r^{-1} \left( r f_r \right)_r + r^{-2} f_{\theta \theta}$. In $3D: r^{-2} \left( r^2 f_{r} \right)_r + r^{-2} \sin^{-2}(\theta) f_{\phi \phi} + r^{-2} \sin^{-1}(\theta) \left(  \sin(\theta) f_{\theta}\right)_\theta$
\textbf{Green's f'n Circle:} For $G=0|_{\partial D}$ we have $G = \frac{-1}{4 \pi} \left( \frac{1}{|x -\xi|} - \frac{1}{|\xi| |x-\xi'|}   \right)$
\textbf{Riemann:} For $u_{xy} + au_{x} + bu_y+cu=f$ we have $\int_D RLu - uL^*R$ $= \textcolor{fg}{\int_D \partial_x \left( Ru_y + auR \right)+ \partial_y\left( -u R_x + buR \right)} = \textcolor{fg}{\int_{\partial D} dy \left( Ru_y+Rau \right) + dx \left( uR_x - buR \right)}$. Expand over triangle going B-P-A (B at bottom right) $\rightarrow$ need $R_x = bR @y=\eta, R_y = aR@ x=\xi, R(P) = 1, L^*R = 0$. Also ensure IVP on $\int_B^P dy Ru_y \rightarrow Ru|^P_B - \int^P_B dy \: uR_y$.
\textbf{R-H:} Derived via $P_x \psi +Q_y \psi = R\psi \rightarrow \int_D ( P \psi  )_x + \left( Q \psi \right)_y (\textcolor{blue}{ = \int_\Gamma \psi P dy - \psi Q dx})\\ = \int_D P \psi_x + Q \psi_y + R \psi  =\int_{D_1+D_2} P \psi_x +Q \psi_y +R \psi$, where $\int_{D_i} = \int_{D_i} \left( P \psi \right)_x + \left( Q \psi \right)_y+ \psi \left( R - P_x -Q_y \right)$. So $\int_\Gamma \psi P dy - \psi Q dx = \int_{\Gamma + C_1 -C_2} \psi P dy - \psi Q dx$ and so $\int_{C_1+C_2} \psi P dy - \psi Q dx = 0 \rightarrow \textcolor{blue}{ dy/dx = \left[ Q \right]^+_-/\left[ P \right]^+_-}$
\textbf{Canonical:} For $au_{xx}+2bu_{xy}+cu_{yy}=f$, we need \textcolor{blue}{Cauchy-Kowalevski} s.t. first derivs defined: $x':= \frac{dx}{ds} $ s.t. on $\Gamma$ $p_0' = x_0'u_{xx}+ y_0'u_{xy},q_0'=x_0'u_{xy}+y_0'u_{yy}.$ Use these 3, solve det A!=0 s.t. $a y_0'^2 -2 b x_0' y_0'+ c x_0'^2 \neq 0$. Solve quadratic s.t. $b^2>ac \rightarrow h, b^2<ac \rightarrow e, b^2=ac \rightarrow p$. \textcolor{blue}{H:} $\lambda_1, \lambda_2 \rightarrow \xi, \eta$. \textcolor{blue}{E:}  $\lambda = \lambda_R \pm i \lambda_I; \lambda_R \rightarrow \xi, \lambda_I \rightarrow \eta$. \textcolor{blue}{P:} $\lambda_1 \rightarrow \xi$, choose $\eta$ independent e.g. $xy, x^2$.
\textbf{Green's Fn:} \textcolor{blue}{DON'T USE GREENS THM USE NORMALS} For $ u_{xx} + u_{yy} + au_x +bu_y +cu=f $ we have $\int_D GLu-uL^*G = \int_D \left( u_x G \right)_x + \left( u_y G \right)_y - \left( uG_x \right)_x - \left( uG_y \right)_y + \left( auG \right)_x + \left( buG \right)_y = \int_D \grad \cdot \left( u_n G - u G_n \right)$+ $\grad \cdot \left( (a \:  b)^{T} \hat n G u\right) = \int_{\partial D} u_n G - u G_n + (a \:  b)^{T} \hat n G$. NB $\hat n = (dy,-dx)$. \textcolor{blue}{Also note for quarter plane} if we have $G_x(0,y) = 0, G(x,0)=0$ then we have same sign at $\xi_1 = (-x,y)$, opposite sign at $\xi_2 = (x,-y)$, and for the third we reflect $\xi_2$ across $y$ axis so we have an opposite sign to $\xi$ at $\xi_3 = (-x,-y)$.
\textbf{Types:} \textcolor{blue}{Quasi:} Coeffs don't depend on highest order derivs \textcolor{blue}{Semi:} Coeffs depend on $x,y$. 
\textbf{Causality:} For a $n$-dim prob, we have $n$ characteristics. Shock intersects $2n$. $\exists \: k$ outgoing, $2n-k$ ingoing. Also have $n$ R-H relations, so $3n-k$ pieces of info. Unknowns are $n$ components of $\va*{u}$ on both sides of shock \& slope $\Rightarrow 2n+1$ unknowns. We demand $3n-k = 2n+1$ so $k=n-1$ outgoing characterisitcs.
\\ \textcolor{red}{SAM:} \textbf{Dists:} Need linearity and continuity: $\exists N,C$ s.t. $| (u, \phi)| \leq C \sum_{m\leq N} \max_{\in [-X,X]} | \phi^{(m)}| $. OR $\lim_{n\rightarrow \infty} (u,\phi_n) = (u, \lim_{n\rightarrow \infty} \phi_n)$ for a sequence $\phi_n \rightarrow 0$ as $n \rightarrow \infty$.
\textbf{Orthog:} $\int_0^\pi \sin(kx)\sin(jx) = \frac{\pi}{2} \delta_{kj}$, same for $\cos$.
\end{document}
