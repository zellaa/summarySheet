\documentclass[a4paper,10pt]{article}
%\documentclass[a4paper,10pt,landscape]{article}
\usepackage[top=2.5cm,bottom=2.5cm,left=2.5cm,right=2.5cm,showframe]{geometry}
\usepackage{xcolor,fancyhdr}
\usepackage{tikz}
\usepackage{amsmath,amssymb,amsthm,amsfonts,physics}
\usepackage{stmaryrd}
\include{stylefile}

\usepackage{lineno}
\linenumbers
\setpagewiselinenumbers

%%%%%%%%%%%%%  PLEASE DO NOT EDIT ANY OF THE LINES ABOVE %%%%%%%%%%%%%%%
% Insert your text between "\begin{document}" and "\end{document}" below. 
% The total length of your summary notes should not exceed 2 sides of a
% single sheet of A4, with maximum 58 lines of text per page.
%%%%%%%%%%%%%%%%%%%%%%%%%%%%%%%%%%%%%%%%%%%%%%%%%%%%%%%%%%%%%%%%%%%%%%%%
\DeclareMathOperator*{\argmin}{arg\,min}
\definecolor{fg}{RGB}{34,139,34}
\begin{document} \noindent
\textcolor{red}{NLA:}
\textbf{Golub} \textcolor{fg}{for $k=1:m,n$:} $u_k = (sgn(b_{k,k}) \norm{b_{k:m,k}}e_1 + b_{k:m,k}$); $u_k := \hat u_k$; $U_k := I-2u_k u_k^T$; $B_{k:m,k:n} := U_k B_{k:m,k:n}; U = [I_{k-1,k-1},0;0,U_k];$\textcolor{blue}{for $j=1:m,n-1$:} $v_k^T := sgn(b_{k,k+1}) \norm{b_{k,k+1:n}} e_1+ b_{k:m,k}$;$V_k := I - 2v_kv_k^T; B_{1:m,k+1:n} = B_{1:m,k+1:n} V_k$; $V = [I_{k,k},0;0,V_k]$\textcolor{blue}{endfor} \textcolor{fg}{endfor}; $2 \cdot (2mn^2 - 2 n^3/3)$
\textbf{Householder} \textcolor{blue}{for $k=[1,n]:$} $x=A_{k:m,k}; v_k = sgn(x) \norm{x}e_k+x;v_k= \frac{v_k}{\norm{v_k}} $ \textcolor{fg}{for $j=[k,n]$} $A_{k:m,j} = A_{k:m,j} - 2v_k\left[ v_k^* A_{k:m,j} \right]$ \textcolor{fg}{endfor} \textcolor{blue}{endfor}. $2mn^2 - \frac{2n^3}{3} $. 
\textbf{MG-S} $V=A;$\textcolor{blue}{for $i=[1,n]:$} $r_{ii}=\norm{v_i}; q_i = \frac{v_i}{r_{ii}}$;\textcolor{fg}{for $j=[i+1,n]$} $ v_j = v_j - ( q_i^T v_j )q_i; r_{ij} = q_i^Tv_j $ \textcolor{fg}{endfor} \textcolor{blue}{endfor}. $ 2mn^2$. 
\textbf{Arnoldi:} $q_1 := \hat b; q_{k+1} h_{k+1,k} = \textcolor{fg}{Aq_k - \sum_{i=1}^k q_i h_{ik}}$; $h_{ik} = q_i^T (A q_k)$; $h_{k+1,k} := \textcolor{fg}{\norm{ v}} \rightarrow AQ_k := Q_k H_k + q_{k+1} [0 \ldots h_{k+1,k}]$.
\textbf{Givens} $3mn^2$
\textbf{SVD:} $= \sum_i^{r:= \min{m,n}} u_i \sigma_i v_i^T$.
\textbf{QR Algo:} $A_{k+1} = Q_k^TA_kQ_k \rightarrow A_{k+1} = \left( Q^{(k)} \right)^T A Q^{(k)} \& A^k = (Q_1 \ldots Q_k) (R_k \ldots R_1) := Q^{(k)}R^{(k)}$, via induction
\textbf{GMRES:} $\min \norm{AQ_k y -b} \rightarrow \min \norm{H_k y - \norm{b}e_1}$
\textbf{CG Convergence:} $\norm{e_k} = \min_{p(0)=1} \norm{p_k(A)e_0}= \min_{p_k(A)} \max \abs{p_k(\lambda)} \norm{e_0} \rightarrow \leq 2\left( (\sqrt{k_2} -1) /(\sqrt{k_2} +1)\right)^k$; need $\alpha := 2(\lambda_1 + \lambda_2)$
\textbf{Cheb:} $T_k(x) = \frac{1}{2} (z^k + z^{-k}); 2xT_k = T_{k+1} + T_{k-1}$
\textbf{MP:} $\sigma(G) \in [\sqrt{m}- \sqrt{n},\sqrt{m}+\sqrt{n}] \rightarrow k_2 = O(1)$
\textbf{Sketch:} with $ GA \hat x= Gb$, and via $C-F$ $\norm {G [A,b] [v,-1]^T} \leq (s+\sqrt{n+1}) \norm{R [v,-1]^T}$,
similar for lower bound via MP 
$\rightarrow \norm{A \hat x-b}  \leq (\sqrt{s} + \sqrt{n+1})/(\sqrt{s} - \sqrt{n+1}) \norm{Ax-b}$
\textbf{Blend:} solve $\norm{(A\hat R^{-1}) y-b}=0$ via CG;$k_2(A\hat R^{-1}) = O(1)$ with $GA = \hat Q\hat R$ \textcolor{fg}{PROOF:}
$A=QR; GA = GQR = \hat G R$. Let $\hat G = \hat Q \hat R$ so $G A = \hat Q \hat R R \rightarrow \tilde R^{-1} = R^{-1} \hat R^{-1} \rightarrow k_2(A \tilde R^{-1}) = k_2(\hat R^{-1}) = O(1)$ by MP. $O(mn)$ to solve via normal
\textbf{Bounds:} $\norm{ABB^{-1}} \geq \norm{AB} \norm{B^{-1}} \rightarrow \norm{A}/\norm{B^{-1}}\geq \norm{AB}$.
\textbf{Weyls:} $\sigma_i(A+B) = \sigma_i(A) + [-\norm{B},\norm{B}]$
\textbf{Rev $\Delta$ Ineq:} $\norm{A-B} \geq \abs{ \norm{A}-\norm{B} }$
\textbf{Courant Application:} $\sigma_i\left( [A_1;A_2] \right) \geq \max\left( \sigma_i (A_1), \sigma_i(A_2) \right)$
\textbf{Schur:} Take $A v_1 = \lambda_1 v_1$; construct $U_1 = [v_1, V_\perp] \rightarrow A U_1 = U_1 [e_1,X]$. Repeat.
\textbf{Conditioning} $\kappa_2(A) = \sigma_1 /\sigma_n = \norm{A}\norm{A^{-1}}$
\textbf{Similarity:} $A \rightarrow P^{-1}AP$, same $\lambda$.
\\ \textcolor{red}{CO:} 
\textbf{SD:} $\norm{x_{k+1}-x_*} \leq \left( k_2(H)-1 \right)/(k_2(H)+1) \norm{x_k-x_*}$ with $H$ hessian
\textbf{bArm:} w/ $\phi(\alpha) = f(x_k + \alpha_k s_k), \psi(\alpha) = \phi(\alpha) - \phi (0)- \beta \alpha \phi'(0) \leq 0$, show $\psi'(0) = (1-\beta) \phi'(0) \leq 0 \rightarrow \psi(\alpha) \downarrow$ with $\alpha$.
\textbf{BFGS:} To show $H_{k+1} \geq 0$ nec. $\gamma^T \delta > 0$. Suff via $\gamma, \delta$ LI $\rightarrow $ use $\norm{\cdot}_H \rightarrow \gamma^T \delta > 0$.
\textbf{Pen. Meth} With $y = -c/\sigma, \norm{\grad_\sigma \Phi} \leq \epsilon^k, \sigma^k \rightarrow 0,x\rightarrow x_*, \grad c(x_*)$ LI, then $y \rightarrow y_*$, $x\rightarrow KKT$. \textcolor{fg}{PROOF:} If $y_* := J^\dagger_* \grad f_* \rightarrow \norm{y_k - y_*} = \norm{J^\dagger_k \grad f_k - I y_*} \leq \norm{J^\dagger_k}\norm{\grad_\sigma \Phi} \rightarrow 0 $. Also, $\grad f_* - J^T_* y_* = 0$, and $c_{k\rightarrow *} = - \sigma^{k\rightarrow *} y_{k \rightarrow *} = 0$ so $x_* \rightarrow KKT$
\textbf{Pen. Meth Newt} Have $w = (J \Delta x + c)/\sigma$ so $[\grad^2 f, J^T;J,-\sigma I] [\Delta x, w]^T = -[\grad f,c]$
\textbf{Trust Region Radius:} $\rho_k := (f(x_k) - f(x_k+ s_k))/(f(x_k) - m_k(s_k))$
\textbf{TR-Method:} If $\rho \approx 1$ then double radius, update step $x_{k+1} = x_k + s_k$. If $\rho \geq 0.1$ then same radius, update step. If $\rho$ small shrink radius, don't update step.
\textbf{Cauchy:} Want $m_k(s_k) \leq m_k(s_{kc})$, where \textcolor{fg}{$s_{kc} := - \alpha_{kc} \grad f(x_k)$}, and \textcolor{fg}{$\alpha_{kc}:= \argmin m_k\left( \alpha \grad f(x_k) \right)$} subject to $\norm{\alpha \grad f} \leq \Delta$, i.e. $\alpha_{max} := \Delta/\norm{ \grad f}$.
\textbf{Calculation of Cauchy:}
We want to prove cauchy model decrease i.e. \textcolor{fg}{$ f(x_k) - m_k(s_k) \geq f(x_k) - m_k (s_{kc}) \geq 0.5 \norm{\grad f_k} \min\left\{ \Delta_k, \frac{\norm{\grad f_k}}{\norm{\grad ^2 f_k}}  \right\}$}.
First define $\Psi (\alpha) := m_k(- \alpha\grad f )$ s.t. $\Psi := f_k - \alpha \norm {f_k}^2 - 0.5 \alpha^2 H_k$, with $H_k := \left[ \grad f_k \right]^T \left[ \grad^2 f_k \right] \left[ \grad f_k \right]$. 
N.B. that \textcolor{blue}{$\alpha_{min}:= \frac{\norm{\grad f_k}^2}{H_k} $} if $H_k  >0$, from $\Psi'(0) <0$.
Now \textcolor{blue}{A: If $H_k \leq 0$} 
then we have $\Psi (\alpha) \leq f_k - \alpha \norm{\grad f_k}^2 \rightarrow \alpha_{kc} = \alpha_{max}$. So, we have $f_k - m_{s_k} \geq f_k - m_{s_{kc}} \geq \norm{\grad f_k} \Delta_k \geq 0.5 \norm{\grad f_k} \min \left\{ \Delta_k \right\}$.
Now \textcolor{blue}{B.i: If $H_k > 0 \rightarrow \alpha_{kc} =\alpha_{min}$}. 
Here $f_k - m_{s_{kc}} = \alpha_{kc} \norm{\grad f}^2 - 0.5 \alpha_{kc}^2 H_k = \frac{\norm{\grad f}^4}{2 H_k} \geq \frac{\norm{\grad f}}{2} \min\left\{ \frac{\norm{\grad f}}{\norm{\grad^2 f}}  \right\}$ via C-S.
Now \textcolor{blue}{B.ii: If $H_k > 0 \rightarrow \alpha_{kc} =\alpha_{max}$}. 
Here $\Delta / \norm{\grad f} \leq \norm{ \grad f}^2 /H_k \rightarrow \alpha_{kc} H_k \leq \norm{\grad f}^2$. So $f_k - m_{kc} = -\alpha_{kc} \norm{\grad f}^2 + \frac{\alpha_{kc}^2}{2} H_k \geq \frac{\norm{\grad f}^2}{2} \alpha_{kc} \geq 0.5 \norm{\grad f} \min\left\{ \Delta_k \right\}$
\textbf{TR-Global Convergence:} If $m_k(s_k) \leq m_k (s_{kc})$ then either $\exists k \geq 0$ s.t. $\grad f_k =0$ or $\lim \norm{\grad f} \rightarrow 0$. Further, require $f \in C^2$, bounded below and also $\grad f$ L-cont. 
\textcolor{blue}{PROOF:} Using def of $\rho$, $f_k - f_{k+1} \geq \frac{0.1}{2} \norm{\grad f_k} \min\left\{ \ldots \right\}$ from above. Let $\norm{ \grad ^2 f} := L$, and assuming $\norm{\grad f} \geq \epsilon$ we have $f_k - f_{k+1} \geq 0.05 \frac{c}{L} \epsilon^2$ assuming TR has a lower bound $c\epsilon/L$. Then sum over all successful jumps s.t. $f_0 - f_{lower} \geq \sum_{i\in \mathbb{S}} f_i - f_{i+1} \geq \abs{\mathbb{S}} \frac{0.05c\epsilon^2}{L} $
\textbf{G-N:} $ \va*{x}_{k+1} = \va*{x}_k - \frac{\grad f_k}{J^T J} $
\textbf{KKT 2nd Order} If we have $\min f$ with $c(x) \geq 0$, $2^{nd}$ order conditions are that $s^T \grad ^2 \mathcal{L} s \geq 0$ for all $s \in \mathcal{A}$, with $\mathcal{A}$ defined s.t: \textcolor{fg}{EITHER} 
$s^T J_i =0 \; \forall \; i$ s.t. $\lambda_i > 0, c_i =0$, 
\textcolor{fg}{OR} 
$s^T J_i \geq 0 \; \forall \; i$ s.t. $\lambda_i = 0, c_i =0$, for $J,c,\lambda$ evaluated at $x_*$
\end{document}
