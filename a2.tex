\documentclass[a4paper,10pt]{article}
%\documentclass[a4paper,10pt,landscape]{article}
\usepackage[top=2.5cm,bottom=2.5cm,left=2.5cm,right=2.5cm,showframe]{geometry}
\usepackage{xcolor,fancyhdr}
\usepackage{tikz}
\usepackage{amsmath,amssymb,amsthm,amsfonts,physics}
\usepackage{stmaryrd}
\include{stylefile}

\usepackage{lineno}
\linenumbers
\setpagewiselinenumbers

%%%%%%%%%%%%%  PLEASE DO NOT EDIT ANY OF THE LINES ABOVE %%%%%%%%%%%%%%%
% Insert your text between "\begin{document}" and "\end{document}" below. 
% The total length of your summary notes should not exceed 2 sides of a
% single sheet of A4, with maximum 58 lines of text per page.
%%%%%%%%%%%%%%%%%%%%%%%%%%%%%%%%%%%%%%%%%%%%%%%%%%%%%%%%%%%%%%%%%%%%%%%%
\usepackage{cancel}
\definecolor{fg}{RGB}{34,139,34}
\begin{document} \noindent
\textcolor{red}{NS:}
\textbf{Inverse $2 \times 2$:} For $A:=[a,b;c,d], A^{-1} := \frac{1}{ad-bc} [d,-b;-c,a]$  
\textbf{Adj $A$:} Adj($A$) is $A^{-1} *$det($A$)
\textbf{Radial:} $r \dot r = x \dot x + y \dot y$, $\dot \theta  =  \left[ \tan ^{-1}\left( y/x \right) \right]' = \frac{x \dot y - \dot x y}{x^2 + y^2} $
\textbf{Classifications:} 
\textcolor{fg}{Node:} $\lambda_i \in \mathbb{R},\Pi \lambda_i > 0$ 
\textcolor{fg}{Centre:} $\lambda_i = \pm i b$
\textcolor{fg}{Focus:} $\lambda_i = a\pm i b$
\textcolor{fg}{Hyperbolic:} Re($\lambda$) $\neq0 \rightarrow$ hyperbolic. \textcolor{blue}{If all $\lambda <0$ for Spec($Df(x_0)$) then A-Stable}
\textbf{Invariant Set:} $\phi_t(S) \subseteq S \; \forall \; t$ 
\textbf{Lim Pts:} $\omega$ pt. if $\lim_{t\rightarrow \infty } \phi(x) = p$, i.e. flows \textit{tend to} $p$. $\alpha$ pt. if $\lim_{t\rightarrow -\infty } \phi(x) = p$.
\textbf{Attracting Set:} A set $A \subseteq S$ if $\exists$ neighbourhood $U$ s.t. $\phi(U) \subseteq U \forall t \geq 0$, and $A = \cap_{t>0} \phi(U)$
\textbf{Dense Orbits:} If $\forall \; \epsilon > 0, x \in A$ with $A$ an attracting set, $\exists \tilde x \in \Gamma s.t. |x -\tilde x| < \epsilon$. I.e. a dense orbit goes as close as needed to \textit{any} point within $A$
\textbf{Attractor:} An attracting set with a dense orbit.
\textbf{Lyapunov Stable:} If $\forall \epsilon > 0, \exists \; \delta > 0 s.t. \forall \; x \in B_\delta, t \geq 0, \phi(t) \in B_\delta$ (i.e. points stay close within region).
\textbf{Asymptotically Stable:} If L-Stable and $\exists \; \delta > 0 s.t. \phi(x) \rightarrow x_0 \forall x \in B_\delta$
\textbf{Lyapunov F'n:} $V(x_0)=0, V(x) > 0 \forall \; x \neq x_0$. 
Then if $\dot V < 0 \rightarrow$ A-Stable, or if $\dot V \leq 0$ L-Stable.
\textbf{Stable Manifold:} If spectrum of $Df(x_0)$ has $k$ eigvals with positive real parts, and $n-k$ with negative, then $\exists$ an $n-k$ dim manifold tangent to $E^s$ s.t. for all $t>0$ $\phi(W^s_{loc}) \subseteq W^s_{loc}$, and $\forall x \in W^s_{loc} \phi(x) \rightarrow x_0$ as time increases. Repeat for $k$-dim unstable manifold but for negative time. Then, define e.g. global stable manifold by $W^s(x_0) := \cup_{t\leq 0}\phi_t\left( W^s_{loc} \right)$. \textcolor{fg}{Note that we search backwards in time for stable, and forwards for unstable!}
\textbf{Centre Manifold:} If $x_0$ not hyperbolic ($0$ real part), then $E^c$ is the centre subspace. Then $\exists W^c$ parallel to $E^c$, of class $C^r$, and invariant under flow. Want bifurcation at $\mu=0$, so with change of variables first find eigvecs $v_1,v_2$. 
Then, construct $P:= [v_1, v_2]$ s.t. $\va*{x} = P \va*{\xi}$. 
\textcolor{fg}{NOTE: first $v_i$ in $P$ is \textit{always} associated with Re($\lambda$)$=0$.}
Solve for $\va*{\xi}$ and then expand with $\eta  = h(\xi, \tilde \mu)$
\textbf{Alt. Centre Manifold:} If vector $v_1 \sim E^c= [a,b]^T$ then we have $y=bx/a$ (e.g. $[1,1]^T \rightarrow y=x$. If bifurcation at $\mu = \alpha$ then have $\mu = \tilde \mu + \alpha $ s.t. bifurcation when $\tilde \mu =0$. Then have $\dot x(x,y,\tilde \mu) = \ldots$ etc.
Next, set up $y=h(x,\tilde \mu) = bx/a + b_1 \tilde \mu + b_2 \tilde \mu^2 + a_2 x^2 + c_2 \tilde \mu x$ and proceed as usual but at $\tilde \mu =0$, s.t. $y$ is along $E^c$.
\textbf{Transcritical Bifurcation:} Always two points, change type at origin. E.g. $\dot x  =  \mu x - x^2$
\textbf{Saddle-Node:} E.g. $\dot x = \mu -x^2$ Bifurcation begins to exist at origin.
\textbf{Supercritical:} E.g. $\dot x = \mu x - x^3$, where stable$\rightarrow 2\times $ stable and one unstable. 
\textbf{Subcritical:} E.g. $\dot x = -\mu x + x^3$, where unstable$\rightarrow 2\times $ unstable and one stable. 
\textbf{General co-dim 1:} If $\dot x = f$ then $\dot x = \mu f_u + 0.5 x^2 f_{xx} + x\mu f_{x \mu} + 0.5 \mu^2 f_{\mu \mu}$. Generally this is a saddle-node but if $f_u = 0 $ we have $\dot x = x\mu f_{x\mu}+ 0.5 x^2 f_{xx}$, which is a transcritical. However if flows invariant under $x = -x$ (reflectional symmetry) then $\dot x = x(\mu f_{x \mu } + \ldots) + x^3 (f_{xxx}/6 + \ldots) \rightarrow $ pitchfork. \textcolor{fg}{Saddle-node stable under perturbations!}
\textbf{Homoclinic Orbits} Sum of roots of cubic = - coeff. of $x^2$
\\ \textcolor{red}{FMM:} 
\textbf{Point Constraint:} If $G(y,z) = 0$ then $\int_a^b \left( \left[ F_y - \frac{d}{dx} \left(F_{y'}\right) \right]\eta + \left[ F_z - \frac{d}{dx} \left( F_{z'} \right) \right] \xi \right)=0$.
Taylor expand $G$ s.t. $G_y \eta + G_z \xi=0$, multiply by $\lambda(x)$ s.t. $\int \lambda G_y \eta + \lambda G_z \xi = 0$.
Rearrange from before s.t. $F_y - \lambda G_y = \frac{d}{dx} \left( F_{y'} \right)$, and similar for $z,z'$.
\textbf{Integral Constraint} If $J[y] = \int F dx$ with $\int G dx = C$ then $\tilde J[y] = \int F- \lambda G dx$ 
\textbf{Hamiltonian:} $H:= y' F_{y'} -F \rightarrow H'=-F_x$. If $F=F(y,\dot y)$ then $H=C$
\textbf{Hamilton's Eqs:} $p := F_{y'}, q=y$ and so $p'=-H_q, q'=H_p$
\textbf{Derive Hamilton Eq:} Have $H = py'-F$, and note $p' = F_y$. 
EQ1:
So $\textcolor{fg}{H_{y'}} = p + y' p_{y'} -F_{y'} = y' \textcolor{blue}{p_{y'}}$.
Also, $\textcolor{fg}{H_{y'}} = H_q \bcancel{q_{y'}} + H_p p_{y'} = H_p \textcolor{blue}{p_{y'}}$ as $q = y$. 
Therefore $y' = \textcolor{blue}{q' = H_p}$.
EQ2: $p' = F_y  = (py'-H)_y = \textcolor{blue}{y'p_y} -\textcolor{fg}{H_y}$.
But $\textcolor{fg}{H_y} = H_p p_y + H_q q_y = \textcolor{blue}{y' p_y} + H_q$.
So finally $\textcolor{blue}{p' = -H_q}$.
\textbf{Free Boundary:} $J[y,b] = \int_a^b F(x,y,y') dx$ where $b$ free. Expand with $y + \epsilon \eta, b + \epsilon \beta \rightarrow J = J_0 + \epsilon \left\{ \int_a^b \eta F_y + \eta' F_{y'} dx + \beta F(b,y(b),y'(b))  \right\}$
If $y(b) = d \rightarrow d = y(b+\epsilon \beta) + \epsilon \eta(b + \epsilon \beta)= y(b) + \epsilon(\beta y'(b) + \eta (b))$ so $\eta(b) = - \beta y'(b)$. IVP on integral so $\beta\left[ F- y'F_{y'} \right]_{x=b} + \int\left( \ldots \right)=0$ so \textcolor{fg}{$F= y'F_{y'}$} at free boundary.
\textbf{Multiple Ind. Variables:} 
For $J = \int F(x,\phi, \phi_x,\phi_y)$.
$J[\phi + \epsilon \eta] = J_0 + \epsilon \int \int_D \left( \eta F_\phi + \eta_x F_{\phi_x} + \eta_y F_{\phi_y} \right)$. Via Green's $\grad \cdot (\eta \va*{f}) = \grad \eta \cdot \va*{f} + \eta \grad \cdot \va*{f} $, so with $ \va*{f} = (F_{\phi_x}, F_{\phi_y})$ we have $\int \int (\eta_x F_{\phi_x} + \eta_y F_{\phi_y}) = - \int \int_D \left( \eta \partial_x\left( F_{\phi_x} \right) + \eta \partial_y \left( F_{\phi_y} \right) \right) + \cancel{\int_{\partial D} \left( \eta \left[ F_{\phi_x} \eta_x + F_{\phi_y} \eta_y \right] \right)}$ .
I.e. we have $F_\phi = \partial_x \left( F_{\phi_x}  \right)+ \partial_y \left( F_{\phi_y} \right)$
\textbf{Control:} Have $\int \xi h_x + \eta h_u dt =0, \dot \xi = \xi  f_x + \eta f_u$. Sub for $\eta$, IVP s.t. $ \frac{d}{dt} \frac{h_u}{f_u}  = h_x - f_x  \frac{h_u}{f_u}   $ and $\dot x = f$
\textbf{Hamiltonian (Control):} $H := f \frac{h_u}{f_u} -h$ s.t. $\dot H = \frac{h_u}{f_u} f_t - h_t \rightarrow$ autonomous if $h_t = f_t = 0$.
\textbf{Fredholm Alt Integ Eqs.} For $y = f+ \int K(x,t)y(t) dt$ we have 
\textcolor{fg}{ONE} (N) has a unique sol $y=0$ if $f=0$, and adjoint has unique sol, or \textcolor{fg}{TWO} (H) as sols $y_1 \ldots y_r$iff $\forall$ solutions to $H^*$, $z_i$, we have $<f,z_i> = 0$. 
\textcolor{fg}{GENERAL CASE:} Have $y = f + \lambda A G_1 + \lambda B G_2$. Solve for system $[\alpha_1, \alpha_2;,\alpha_3,\alpha_4][A,B]^T = [\gamma_1,\gamma_2]^T$ with NONUNIQUE sols for $\lambda = \lambda_*$. 
Now for $\lambda= \lambda_*$, want to solve $L^*w = 0$ and show this is orthogonal to RHS. First solve $[\alpha_1, \alpha_2;,\alpha_3,\alpha_4][A,B]^T = [0,0]^T$. Then we have $w = \lambda_* A (F(G_1,G_2))$. Check if $\int f w = 0$. If so, return to NONHOM case and solve $[\alpha_1, \alpha_2;,\alpha_3,\alpha_4]_{\lambda_*}[A,B]^T = [\gamma_1,\gamma_2]^T$ to get $B = -\frac{\alpha_1}{\alpha_2} A + \frac{\gamma_1}{\alpha_2}$. Sub this into $y = f + \lambda_* A G_1 + \lambda_* B(A) G_2$.
\textcolor{blue}{EX:} 
Solve $y=1-x^2 + \lambda \int (1-5 x^2 t^2)y(t)\; dt = 1-x^2 + \lambda A - 5 \lambda B x^2$.
Have $A:= \int y_N(t) = \int 1-x^2 + \lambda A + \ldots = \lambda A - \frac{5 \lambda}{3} + \frac{2}{3}$. Repeat for $B$ s.t. $[1-\lambda, 5\lambda/3; -\lambda/3, 1+\lambda] [A,B]^T = [2/3, 2/15]^T$.
Unique sols if $\lambda \neq \pm \frac{3}{2} \rightarrow$ try when $\lambda_* = \frac{-3}{2} $.
Have $L^* w_H =  \lambda A - 5 \lambda_* B x^2 \rightarrow A := \int \lambda_* A - 5 \lambda_* B x^2$, and $B:= \int \ldots$. Both give consistent results $A = B$ so $w_H = \lambda_*A(1-5x^2)$. Check that $\int w_H(x) (1-x^2) = 0$, so we have shown nullspace of adj. orthog. to RHS. 
\textcolor{blue}{Note that we may also find $A,B$ for adjoint quicker via $[1-\lambda, 5\lambda/3; -\lambda/3, 1+\lambda]_{\lambda_*} [A,B]^T =[0,0]^T$.}
Lastly, return to (N), and having verified $\lambda_*$ permits a solution, solve $[1-\lambda, 5\lambda/3; -\lambda/3, 1+\lambda]_{\lambda_*} [A,B]^T =[2/3,2/15]^T \rightarrow A-B = 4/15$ when $\lambda = -3/2$. Sub this into $y=1-x^2\ldots$ for solution.
\textbf{Fred Diff Eq.} For nonunique sol to exist, need $<Ly,w> = <f,w> \forall \; w s.t. L^*w=0$
\textbf{Trig:} $\int^{2 \pi}_0 \cos^2 = \int^{2 \pi}_0 \sin^2 = \pi$
\\ \textcolor{red}{FPDE:}
\textbf{Types:} $1^{st}: \exists$ scale s.t. solution found, not so for $2^{nd}$.
\textbf{Heat:} $\hat T = u(\hat T_{\infty} - \hat T_{-\infty}) + \hat T_{-\infty}$
\textbf{Oil Spread:} Dims: $x = x_f + \varepsilon \xi, t = \tau$
\textbf{Ground Spread:} $(1-s) \phi h_t + Q_x =0; Q\sim -h h_x, 0<x_s<x_f$. Have $h(x_f)=0, h_t(x_s)=0$, and $h h_x |_{x=0,x_f} = 0$ (i.e. no flux at centre and front), and $h,hh_x$ cont. at joint.
\textbf{Expansions:} Let $\xi = z + \epsilon \eta$ for perturbations
\textbf{Scale:} Try $x=x_f + \epsilon \xi$ for groundwater
\textbf{Stefan:} $S_0 = C\left( T_1-T_m \right)/L$, condition = $\rho L \dot s = k T_x |^{s_+}_{s_-}$
\textbf{1ph Stefan:} Bar = $T_h | liq |_s sol | INS$. 
Use $T = T_m + (T_1-T_m)u$ s.t. $S_0 u_t = u_{xx}, u=1\;@ \; x=0,\left\{\dot s = -u_x,u=0\right\} @ \;x=s, s(0)=0$. Sim. sol is $s= \beta \sqrt{t}, f=f(x/\sqrt{t})$
\textbf{2ph Stefan:} (melting)
Use $T = T_m + (T_1-T_m)u$ s.t.
$S_0 u_t = u_{xx} \; @ \; 0<x<s,(S_0/\kappa) u_t = u_{xx} \; @ \; s<x<1, u=1\;@ \; x=0,u_x=0 \;@ \; x=1,\left\{\dot s =Ku_x|_{s_+} -u_x|_{s_-},u=0\right\} @ \;x=s, \left\{s=0, u=-\theta\right\} \; @ \; x=0$. Here $\theta := (T_m-T_0)/(T_1-T_m), \kappa := c_1k_1/(c_2k_2), K:= k_2/k_1$ Sim. sol is $s= \beta \sqrt{t}, f=f(x/\sqrt{t})$
\textbf{2-Dim:} Normal velocity $ U_n := \hat n \cdot u = K(u_2)_n - (u_1)_n$ in Stefan Prob. If $x=f(y,t)$ then $\hat n := \grad (x-f) = [1,-f_y]^T/\sqrt{1+f_y^2}$.
\textcolor{blue}{EX:}
Consider $u_1 :=- \lambda_1 (x-V_0 t) + \epsilon \tilde u_1(x,y,t),  u_2 :=- \lambda_2 (x-V_0 t) + \epsilon \tilde u_2(x,y,t)$.
If position of boundary $x_b := V_0 t + \epsilon \xi(y,t) = f(y,t)$.
So, normal velocity $U_n:= \hat n \cdot u$, where we have $u = [\dot x_b, \dot y_b]^T$ so $U_n \sim [1,-f_y] [\dot x_b, 0]^T = \dot x_b = V_0 + \epsilon \xi_t$. On RHS, we have $(u_i)_n = \hat n \dot \grad u_i \sim [1,-f_y]^T \grad u = [1, - \epsilon \xi_y] [u_x,u_y]^2 = u_x - \epsilon \xi_y u_y$. For e.g. $u_1$ we have $(u_1)_n = -\lambda_1 + \epsilon (\tilde u_1)_x - \epsilon^2 \xi_y ( \tilde u_1)_y$. 
So, to $O(\epsilon^0)$ we have $V_0 = -\lambda_2 K + \lambda_1$, and to $O(\epsilon^1)$ $\xi_t = K ( \tilde u_2)_x - (\tilde u_1)_x$.
\textbf{Welding:} Have $0<s_2<s_1$. Have cold $x=a$, no flux $x=0$. $\theta=1$ in liquid. In mush $\rho L \theta_t = J^2/\sigma$, CoE $\rightarrow \theta \rho L \dot s + k T_x |^{s_+}_{s_-} = 0$. Have $\theta$ cont. $(=0)$ at $s_1$. 
I.e. we have $S_0 u_t = u_{xx}+q, u_x = 0 \; @ \; x=0, u=-1 \; @ \; x=1, \theta =0 \;@ \; x=s_1$. Also $\theta_t = q$ in mush.
\textbf{ERF:} $\erf{x} = \frac{2}{\sqrt{\pi}} \int_0^y e^{-y^2}$ s.t. if $f' = e^{ -\frac{\eta^4}{4k} }, f:= A \sqrt{\pi k} \erf{ \frac{\eta}{2 k}} + B $, $(\erf{x})' = \frac{2}{\sqrt{\pi}} e^{-y^2}$.
\end{document}
