\documentclass[a4paper,10pt]{article}
%\documentclass[a4paper,10pt,landscape]{article}
\usepackage[top=2.5cm,bottom=2.5cm,left=2.5cm,right=2.5cm,showframe]{geometry}
\usepackage{xcolor,fancyhdr}
\usepackage{tikz}
\usepackage{amsmath,amssymb,amsthm,amsfonts,physics}
\usepackage{stmaryrd}
\include{stylefile}

\usepackage{lineno}
\linenumbers
\setpagewiselinenumbers

%%%%%%%%%%%%%  PLEASE DO NOT EDIT ANY OF THE LINES ABOVE %%%%%%%%%%%%%%%
% Insert your text between "\begin{document}" and "\end{document}" below. 
% The total length of your summary notes should not exceed 2 sides of a
% single sheet of A4, with maximum 58 lines of text per page.
%%%%%%%%%%%%%%%%%%%%%%%%%%%%%%%%%%%%%%%%%%%%%%%%%%%%%%%%%%%%%%%%%%%%%%%%

\definecolor{fg}{RGB}{34,139,34}
\begin{document} \noindent
\textcolor{red}{NS:}
\textbf{Classifications:} 
\textcolor{fg}{Node:} $\lambda_i \in \mathbb{R},\Pi \lambda_i > 0$ 
\textcolor{fg}{Centre:} $\lambda_i = \pm i b$
\textcolor{fg}{Focus:} $\lambda_i = a\pm i b$
\textcolor{fg}{Hyperbolic:} Re($\lambda$) $\neq0 \rightarrow$ hyperbolic. \textcolor{blue}{If all $\lambda <0$ for Spec($Df(x_0)$) then A-Stable}
\textbf{Invariant Set:} $\phi_t(S) \subseteq S$
\textbf{Lim Pts:} $\omega$ pt. if $\lim_{t\rightarrow \infty } \phi(x) = p$, i.e. flows \textit{tend to} $p$. $\alpha$ pt. if $\lim_{t\rightarrow -\infty } \phi(x) = p$.
\textbf{Attracting Set:} A set $A \subseteq S$ if $\exists$ neighbourhood $U$ s.t. $\phi(U) \subseteq U \forall t \geq 0$, and $A = \cap_{t>0} \phi(U)$
\textbf{Dense Orbits:} If $\forall \; \epsilon > 0, x \in A$ with $A$ an attracting set, $\exists \tilde x \in \Gamma s.t. |x -\tilde x| < \epsilon$. I.e. a dense orbit goes as close as needed to \textit{any} point within $A$
\textbf{Attractor:} An attracting set with a dense orbit.
\textbf{Lyapunov Stable:} If $\forall \epsilon > 0, \exists \; \delta > 0 s.t. \forall \; x \in B_\delta, t \geq 0, \phi(t) \in B_\delta$ (i.e. points stay close within region).
\textbf{Asymptotically Stable:} If L-Stable and $\exists \; \delta > 0 s.t. \phi(x) \rightarrow x_0 \forall x \in B_\delta$
\textbf{Lyapunov F'n:} $V(x_0)=0, V(x) > 0 \forall \; x \neq x_0$. 
Then if $\dot V < 0 \rightarrow$ A-Stable, or if $\dot V \leq 0$ L-Stable.
\textbf{Stable Manifold:} If spectrum of $Df(x_0)$ has $k$ eigvals with positive real parts, and $n-k$ with negative, then $\exists$ an $n-k$ dim manifold tangent to $E^s$ s.t. for all $t>0$ $\phi(W^s_{loc}) \subseteq W^s_{loc}$, and $\forall x \in W^s_{loc} \phi(x) \rightarrow x_0$ as time increases. Repeat for $k$-dim unstable manifold but for negative time. Then, define e.g. global stable manifold by $W^s(x_0) := \cup_{t\leq 0}\phi_t\left( W^s_{loc} \right)$. \textcolor{fg}{Note that we search backwards in time for stable, and forwards for unstable!}
\textbf{Centre Manifold:} If $x_0$ not hyperbolic ($0$ real part), then $E^c$ is the centre subspace. Then $\exists W^c$ parallel to $E^c$, of class $C^r$, and invariant under flow. Want bifurcation at $\mu=0$, so with change of variables first find eigvecs $v_1,v_2$. Then, construct $P:= [v_1, v_2]$ s.t. $\va*{x} = P \va*{\xi}$. Solve for $\va*{\xi}$ and then expand with $\eta  = h(\xi, \tilde \mu)$
\textbf{Transcritical Bifurcation:} Always two points, change type at origin. E.g. $\dot x  =  \mu x - x^2$
\textbf{Saddle-Node:} E.g. $\dot x = \mu -x^2$ Bifurcation begins to exist at origin.
\textbf{Supercritical:} E.g. $\dot x = \mu x - x^3$, where stable$\rightarrow 2\times $ stable and one unstable. 
\textbf{Subcritical:} E.g. $\dot x = -\mu x + x^3$, where unstable$\rightarrow 2\times $ unstable and one stable. 
\textbf{General co-dim 1:} If $\dot x = f$ then $\dot x = \mu f_u + 0.5 x^2 f_{xx} + x\mu f_{x \mu} + 0.5 \mu^2 f_{\mu \mu}$. Generally this is a saddle-node but if $f_u = 0 $ we have $\dot x = x\mu f_{x\mu}+ 0.5 x^2 f_{xx}$, which is a transcritical. However if $x = -x$ then $\dot x = x(\mu f_{x \mu } + \ldots) + x^3 (f_{xxx}/6 + \ldots) \rightarrow $ pitchfork. \textcolor{fg}{Saddle-node stable under perturbations!}
\textbf{HomoClinic Orbits} Sum of roots of cubic = - coeff. of $x^2$
\\ \textcolor{red}{FPDE:}
\textbf{Types:} $1^{st}: \exists$ scale s.t. solution found, not so for $2^{nd}$.
\textbf{Heat:} $\hat T = u(\hat T_{\infty} - \hat T_{-\infty}) + \hat T_{-\infty}$
\textbf{Oil Spread:} Dims: $x = x_f + \varepsilon \xi, t = \tau$
\textbf{Ground Spread:} $(1-s) \phi h_t + Q_x =0; Q\sim -h h_x, 0<x_s<x_f$. Have $h(x_f)=0, h_t(x_s)=0$, and $h h_x |_{x=0,x_f} = 0$ (i.e. no flux at centre and front), and $h,hh_x$ cont. at joint.
\textbf{Expansions:} Let $\xi = z + \epsilon \eta$ for perturbations
\textbf{Scale:} Try $x=x_f + \epsilon \xi$ for groundwater
\textbf{Stefan:} $S_0 = C\left( T_1-T_m \right)/L$, condition = $\rho L \dot s = k T_x |^{s_+}_{s_-}$
\textbf{1ph Stefan:} Bar = $T_h | liq |_s sol | INS$. 
Use $T = T_m + (T_1-T_m)u$ s.t. $S_0 u_t = u_{xx}, u=1\;@ \; x=0,\left\{\dot s = -u_x,u=0\right\} @ \;x=s, s(0)=0$. Sim. sol is $s= \beta \sqrt{t}, f=f(x/\sqrt{t})$
\textbf{2ph Stefan:}
Use $T = T_m + (T_1-T_m)u$ s.t.
$S_0 u_t = u_{xx} \; @ \; 0<x<s,(S_0/\kappa) u_t = u_{xx} \; @ \; s<x<1, u=1\;@ \; x=0,u_x=0 \;@ \; x=1,\left\{\dot s =Ku_x|_{s_+} -u_x|_{s_-},u=0\right\} @ \;x=s, \left\{s=0, u=-\theta\right\} \; @ \; x=0$. Here $\theta := (T_m-T_0)/(T_1-T_m), \kappa := c_1k_1/(c_2k_2), K:= k_2/k_1$ Sim. sol is $s= \beta \sqrt{t}, f=f(x/\sqrt{t})$
\textbf{2-Dim:} $ U_n = \hat n \cdot u = K(u_2)_n - (u_1)_n$. If $x=f(y,t)$ then $\hat n := \grad (x-f) = [1,-f_y]^T/\sqrt{1+f_y^2}$
\textbf{Welding:} Have $0<s_2<s_1$. Have cold $x=a$, no flux $x=0$. $\theta=1$ in liquid. In mush $\rho L \theta_t = J^2/\sigma$, CoE $\rightarrow \theta \rho L \dot s + k T_x |^{s_+}_{s_-} = 0$. Have $\theta$ cont. $(=0)$ at $s_1$. 
I.e. we have $S_0 u_t = u_{xx}+q, u_x = 0 \; @ \; x=0, u=-1 \; @ \; x=1, \theta =0 \;@ \; x=s_1$. Also $\theta_t = q$ in mush.
\\ \textcolor{red}{FMM:} 
\textbf{Integral Constraint} If $J[y] = \int F dx$ with $\int G dx = C$ then $\tilde J[y] = \int F- \lambda G dx$ 
\textbf{Hamiltonian:} $H:= y' F_{y'} -F \rightarrow H'=-F_x$. If $F=F(y,\dot y)$ then $H=C$
\textbf{Hamilton's Eqs:} $p := F_{y'}, q=y$ and so $p'=-H_q, q'=H_p$
\textbf{Free Boundary:} $J[y,b] = \int_a^b F(x,y,y') dx$ where $b$ free. Expand with $y + \epsilon \eta, b + \epsilon \beta \rightarrow J = J_0 + \epsilon \left\{ \int_a^b \eta F_y + \eta' F_{y'} dx + \beta F(b,y(b),y'(b))  \right\}$
If $y(b) = d \rightarrow d = y(b+\epsilon \beta) + \epsilon \eta(b + \epsilon \beta)= y(b) + \epsilon(\beta y'(b) + \eta (b))$ so $\eta(b) = - \beta y'(b)$. IVP on integral so $\beta\left[ F- y'F_{y'} \right]_{x=b} + \int\left( \ldots \right)=0$ so \textcolor{fg}{$F= y'F_{y'}$} at free boundary.
\textbf{Control:} Have $\int \xi h_x + \eta h_u dt =0, \dot \xi = \xi  f_x + \eta f_u$. Sub for $\eta$, IVP s.t. $ \frac{d}{dt} \frac{h_u}{f_u}  = h_x - f_x  \frac{h_u}{f_u}   $ and $\dot x = f$
\textbf{Hamiltonian (Control):} $H := f \frac{h_u}{f_u} -h$ s.t. $\dot H = \frac{h_u}{f_u} f_t - h_t \rightarrow$ autonomous if $h_t = f_t = 0$.
\textbf{Fredholm Alt Integ Eqs.} For $y = f+ \int K(x,t)y(t) dt$ we have 
\textcolor{fg}{ONE} (N) has a unique sol $y=0$ if $f=0$, and adjoint has unique sol, or \textcolor{fg}{TWO} (H) as sols $y_1 \ldots y_r$iff $\forall$ solutions to $H^*$, $z_i$, we have $<f,z_i> = 0$. \textcolor{blue}{EX:} Solve $y=f+\lambda \int \sin(x+t) y(t)dt$. Unique sol iff (H) has trivial sol $\rightarrow X_1 = \int y \cos(t) = \int \cos(t) y_H(t) \rightarrow$ solve $[1, -\lambda \pi; -\lambda \pi, 1] [X_1, X_2]^T = [0,0]^T \rightarrow$ unique sol if $\lambda \neq \pm 1/\pi$. In this case $X_1 = \int \cos(x) y_N(x) =\lambda \pi X_2 + \int f(x) \cos(x)$, and similar for $X_2$. Invert matrix and solve. If non-unique sol, then find sols to (H) first. If $\lambda = 1/\pi$ then $X_1 = X_2 = X$ so $Ly = y- \pi^{-1} \left( \sin (x) +\cos(x) \right) \int \cos(x) y(x) dx$ with sols $y=c_1 \left( \sin(x) + \cos(x) \right)$ by inspection. Problem self adjoint so $Ly = 0 = L^* w$ so need $\int f(x) w(x) =0$ i.e. $\int f(x) (\sin(x) + \cos(x)) = 0$, repeat for $\lambda =-1/\pi$. Then $y = y_p(x) + \sum_i y_{h,i}(x)$
\textbf{Fred Diff Eq.} For nonunique sol to exist, need $<Ly,w> = <f,w> \forall \; w s.t. L^*w=0$
\end{document}
