\documentclass[a4paper,10pt]{article}
%\documentclass[a4paper,10pt,landscape]{article}
\usepackage[top=2.5cm,bottom=2.5cm,left=2.5cm,right=2.5cm,showframe]{geometry}
\usepackage{xcolor,fancyhdr}
\usepackage{tikz}
\usepackage{amsmath,amssymb,amsthm,amsfonts,physics}
\usepackage{stmaryrd}
\include{stylefile}

\usepackage{lineno}
\linenumbers
\setpagewiselinenumbers

%%%%%%%%%%%%%  PLEASE DO NOT EDIT ANY OF THE LINES ABOVE %%%%%%%%%%%%%%%
% Insert your text between "\begin{document}" and "\end{document}" below. 
% The total length of your summary notes should not exceed 2 sides of a
% single sheet of A4, with maximum 58 lines of text per page.
%%%%%%%%%%%%%%%%%%%%%%%%%%%%%%%%%%%%%%%%%%%%%%%%%%%%%%%%%%%%%%%%%%%%%%%%

\definecolor{fg}{RGB}{34,139,34}
\begin{document} \noindent
\textcolor{red}{APDE:} \textbf{Charpit:} $F(p,q,u,x,y)=0$ with $\color{blue}{u_x = p,u_y=q,\dot x= F_p,\dot y=F_q}$. Then via $F_x,F_y, \& \: p_y=q_x \rightarrow \color{blue}{p_\tau = -F_x - pF_u}$, $\color{blue}{q_\tau = -F_y-qF_u}, \color{blue}{u_\tau = pF_p + qF_q}$. Also, $\color{blue}{u0_s = p_0x0_s + q_0y0_s; F_0 = 0}$ - last 2 needed to show $u$ defined on $\Gamma$.
\textbf{Laplacian:} In $2D: r^{-1} \left( r f_r \right)_r + r^{-2} f_{\theta \theta}$. In $3D: r^{-2} \left( r^2 f_{r} \right)_r + r^{-2} \sin^{-2}(\theta) f_{\phi \phi} + r^{-2} \sin^{-1}(\theta) \left(  \sin(\theta) f_{\theta}\right)_\theta$
\textbf{Riemann:} For $u_{xy} + au_{x} + bu_y+cu=f$ we have $\int_D RLu - uL^*R$ $= \textcolor{fg}{\int_D \partial_x \left( Ru_y + auR \right)+ \partial_y\left( -u R_x + buR \right)} = \textcolor{fg}{\int_{\partial D} dy \left( Ru_y+Rau \right) + dx \left( uR_x - buR \right)}$. Expand over triangle going B-P-A (B at bottom right) $\rightarrow$ need $R_x = bR @y=\eta, R_y = aR@ x=\xi, R(P) = 1, L^*R = 0$. Also ensure IVP to get $R_y,R_x$!
\textbf{R-H:} Derived via $P_x \psi +Q_y \psi = R\psi \rightarrow \int_D \left( P_x \psi  \right)_x + \left( Q_y \psi \right)_y (\textcolor{blue}{ = \int_\Gamma \psi P dy - \psi Q dx}) = \int_D P \psi_x + Q \psi_y + R \psi  =\int_{D_1+D_2} P \psi_x +Q \psi_y +R \psi$, where $\int_{D_i} = \int_{D_i} \left( P \psi \right)_x + \left( Q \psi \right)_y+ \psi \left( R - P_x -Q_y \right)$. So $\int_\Gamma \psi P dy - \psi Q dx = \int_{\Gamma + C_1 -C_2} \psi P dy - \psi Q dx$ and so $\int_{C_1+C_2} \psi P dy - \psi Q dx = 0 \rightarrow \textcolor{blue}{ dy/dx = \left[ Q \right]^+_-/\left[ P \right]^+_-}$
\textbf{Canonical:} For $au_{xx}+2bu_{xy}+cu_{yy}=f$, we need \textcolor{blue}{Cauchy-Kowalevski} s.t. first derivs defined: $x':= \frac{dx}{ds} $ s.t. on $\Gamma$ $p_0' = x_0'u_{xx}+ y_0'u_{xy},q_0'=x_0'u_{xy}+y_0'u_{yy}.$ Use these 3, solve det A!=0 s.t. $a y_0'^2 -2 b x_0' y_0'+ c x_0'^2 \neq 0$. Solve quadratic s.t. $b^2>ac \rightarrow h, b^2<ac \rightarrow e, b^2=ac \rightarrow p$. \textcolor{blue}{H:} $\lambda_1, \lambda_2 \rightarrow \xi, \eta$. \textcolor{blue}{E:}  $\lambda = \lambda_R \pm i \lambda_I; \lambda_R \rightarrow \xi, \lambda_I \rightarrow \eta$. \textcolor{blue}{P:} $\lambda_1 \rightarrow \xi$, choose $\eta$ independent e.g. $xy, x^2$.
\textbf{Green's Fn:} For $ u_{xx} + u_{yy} + au_x +bu_y +cu=f $ we have $\int_D GLu-uL^*G = \int_D \left( u_x G \right)_x + \left( u_y G \right)_y - \left( uG_x \right)_x - \left( uG_y \right)_y + \left( auG \right)_x + \left( buG \right)_y = \int_D \grad \cdot \left( u_n G - u G_n \right)$+ $\grad \cdot \left( (a \:  b)^{T} \hat n G u\right) = \int_{\partial D} u_n G - u G_n + (a \:  b)^{T} \hat n G$. NB $\hat n = (dy,-dx)$. \textcolor{blue}{Also note for quarter plane} if we have $G_x(0,y) = 0, G(x,0)=0$ then we have same sign at $\xi_1 = (-x,y)$, opposite sign at $\xi_2 = (x,-y)$, and for the third we reflect $\xi_2$ across $y$ axis so we have an opposite sign to $\xi$ at $\xi_3 = (-x,-y)$.
\textbf{Types:} \textcolor{blue}{Quasi:} Coeffs don't depend on highest order derivs \textcolor{blue}{Semi:} Coeffs depend on $x,y$. 
\textbf{Causality:} For a $n$-dim prob, we have $n$ characteristics. Shock intersects $2n$. $\exists \: k$ outgoing, $2n-k$ ingoing. Also have $n$ R-H relations, so $3n-k$ pieces of info. Unknowns are $n$ components of $\va*{u}$ on both sides of shock \& slope $\Rightarrow 2n+1$ unknowns. We demand $3n-k = 2n+1$ so $k=n-1$ outgoing characterisitcs.
\\ \textcolor{red}{SAM:} \textbf{Dists:} Need linearity and continuity: $\exists N,C$ s.t. $| (u, \phi)| \leq C \sum_{m\leq N} \max_{\in [-X,X]} | \phi^{(m)}| $. OR $\lim_{n\rightarrow \infty} (u,\phi_n) = (u, \lim_{n\rightarrow \infty} \phi_n)$ for a sequence $\phi_n \rightarrow 0$ as $n \rightarrow \infty$.
\textbf{Orthog:} $\int_0^\pi \sin(kx)\sin(jx) = \frac{\pi}{2} \delta_{kj}$, same for $\cos$.
\newpage \noindent
\textcolor{red}{NLA:} \textbf{Cholesky} For matrix $\left[ a_{11},w^*;w,K \right] = R_1^T \left[ I,0;0,K- \frac{w w^*}{a_{11}}  \right]\left[ \alpha, w^*/\alpha;0,I \right]  $ we have a decomp: \textcolor{blue}{for $k=[1,m-1]:$} \textcolor{fg}{for $j=[k+1,m]$} $R_{j,j:m} = R_{j,j:m} - \frac{R_{kj}}{R_{kk}} R_{k,j:m}$ \textcolor{fg}{endfor} $R_{k,k:m} = \frac{R_{k,k:m}}{\sqrt{ R_{kk} }} $ \textcolor{blue}{endfor}. $ \frac{m^3}{3} $. 
\textbf{Householder} \textcolor{blue}{for $k=[1,n]:$} $x=A_{k:m,k}; v_k = sgn(x) \norm{x}e_k+x;v_k= \frac{v_k}{\norm{v_k}} $ \textcolor{fg}{for $j=[k,n]$} $A_{k:m,j} = A_{k:m,j} - 2v_k\left[ v_k^* A_{k:m,j} \right]$ \textcolor{fg}{endfor} \textcolor{blue}{endfor}. $ \frac{2mn^2}{3} $. 
\textbf{LU} $U=A,L=I$ \textcolor{blue}{for $k=[1,m-1]:$} \textcolor{fg}{for $j=[k+1,m]$} $U_{j,k:m} = U_{j,k:m} - \frac{U_{jk}}{U_{kk}} U_{k,k:m}  $ \textcolor{fg}{endfor} \textcolor{blue}{endfor}. $ \frac{2m^3}{3} $. 
\textbf{MG-S} $V=A;$\textcolor{blue}{for $i=[1,n]:$} $r_{ii}=\norm{v_i}; q_i = \frac{v_i}{r_{ii}}$;\textcolor{fg}{for $j=[i+1,n]$} $ v_j = v_j - ( q_i^T v_j )q_i; r_{ij} = q_i^Tv_j $ \textcolor{fg}{endfor} \textcolor{blue}{endfor}. $ 2mn^2$. 
\textbf{Givens} $3mn^2$
\textbf{SVD:} $= \sum_i^{r:= \min{m,n}} u_i \sigma_i v_i^T$.
\textbf{Bounds:} $\norm{ABB^{-1}} \geq \norm{AB} \norm{B^{-1}} \rightarrow \norm{A}/\norm{B^{-1}}\geq \norm{AB}$.
\textbf{Norms:}$\norm{A}_F = \sqrt{\sum_i\left( \sigma_i \right)^2} =\sqrt{Tr\left( AA^T \right)}$, $\norm{A}_\infty=$ max row sum.
\textbf{Low-Rank:} For $A\in \mathbb{R}^{m\times n} \min \norm{A-B} = \norm{A-A_r}$. Proof via $B := B_1 B_2^T$ with $B_1 \in \mathbb{R}^{m\times r}$; $\exists W s.t. B_2^{T} W = 0$ with null($W$)$\geq n-r$. Then $\exists \: x_V,x_W s.t. V_{r+1}x_V = - W x_W$. So $\norm{A-B} = \norm{AW} \geq \norm{A V_{r+1} x_V} \geq \sigma_{r+1}$ For reverse $B:= A_r$
\textbf{Courant:} $\sigma_i = \max_{dim(S)=i}\left\{ \min_{x} \norm{Ax}/\norm{x} \right\}$. Proof via $V_i = [v_i \dots v_n]$, so dim($S$)+dim($V_i$) $=n+1$ so $\exists \: w\in S \cap V_i$. Then $\norm{Aw} \leq \sigma_i$. For reverse take $w=v_i$ when $S=[v_1\dots v_i]$
\textbf{Schur:} Take $A v_1 = \lambda_1 v_1$; construct $U_1 = [v_1, V_\perp] \rightarrow A U_1 = U_1 [e_1,X]$. Repeat.
\textbf{Back Subst:} For $Ux=y$ we have $x_{n-i} = \left( y_{n-i} - \sum_{n-i+1}^n u_{n-i,j}x_j \right)/u_{n-i,n-i}; O(i)$ per iteration so $O(n^2)$ total.
\textbf{Backwards Stable:} When $\hat f(x) = f (x + \Delta x)$ with $\norm{\Delta x}/\norm{x} \leq O(\varepsilon)$
\textbf{Conditioning} $\kappa_2(A) = \sigma_1 /\sigma_n = \norm{A}\norm{A^{-1}}$
\\ \textcolor{red}{NPDE:} \textbf{Hyperbolic:} \textcolor{blue}{Implicit:} $\left( A-B,A \right) = \frac{1}{2}( \norm{A}^2-\norm{B}^2) + \frac{1}{2} \norm{A-B}^2$ (time),$(-D_x^+D_x^-U^{m+1},\\U^{m+1}-U^m) = ( D_x^-U^{m+1} - D_x^- U^m,D_x^-U^{m+1})$ (space). \textcolor{blue}{Explicit:} 1st rewrite in terms of $D_t^{+-} (\Delta t)^{-2}U_j^m + \frac{c^2 (\Delta t)^2}{4} D_x^{+-}((\Delta t)^{-2}U_j^m) -c^2D_x^{+-}\left( U_j^{m+1}+2U_j^m+U_j^{m-1} \right)$. Then use $\color{blue}{ \left( D(A-B),A+B \right)} = (DA,A)-(DB,B);  \left( D(A+B),A-B \right) = (DA,A)-(DB,B) $ by multiplying by $U^{m+1} - U^{m-1}$. Finally WTS $\norm{V_m}^2- \frac{c^2 \left( \Delta t \right)^2}{4} \norm{D_x^- V^m}^2 \geq 0$. Done by noticing: $\norm{D_x^- V^m}^2 = \sum_i^J \Delta x | D_x^- V_j^m|^2 = 1/\Delta x \sum_i^J\\ \left( V_j^m - V_{j-1}^{m} \right)^2 \leq 2/\Delta x \sum_i^J (V_j^m)^2 + (V_{j-1}^{m})^2 = 4/\Delta x^2 \sum_i^{J-1} \Delta x \left( V_j^m \right)^2$
\textbf{Max Principle:} For $- \Delta u = f \leq 0 \rightarrow \max u \in \partial D$. First show contradiction assuming $LU = f < 0$, then try some auxillary function $\psi = U + \alpha\left( T_{\max} \right) g\left( x_i,y_i \right)$ s.t. $L\psi < 0$ so $\max \psi = \max_{\in \partial D} \psi$. Gets $\max e_{i,j}$; change to $-\alpha$ for $\min e_{i,j}$.
\textbf{P-F Ineq:} $\norm{V}^2_h \leq c_\star ||D_x^-V]|^2$
\textbf{Weak Deriv:} $w$ is a weak derivative of $u$ if $\int dx \: w v = \left( -1 \right)^{\abs{\alpha}} \int dx \: u (D^\alpha v)$
\textbf{Parseval:} $\textcolor{blue}{\int dk \: \hat u(k) v(k)} = \int dk \:v(k) \left( \int dx \: u(x) e^{-ixk} \right) = \int dx\:  u(x) \left( \int dk \: v(k) e^{-ixk} \right) = \textcolor{fg}{\int dx \: u(x) \hat v(x)}$. Now $v(k) := \textcolor{blue}{\overline{\hat u(k)}}  = \overline{F\left[ u(k) \right]} = \overline{ \int dk \: u(k) e^{-ixk} } = \int dk \: \overline{u(k)} e^{ixk} = 2\pi F^{-1}\left[ \overline{u(k)} \right] \Rightarrow \textcolor{fg}{\hat v(x) = 2 \pi \overline{u(x)}}$ 
\textbf{Iterative:} If $U^{j+1} = U^j- \tau \left( AU^j - F \right) \rightarrow U-U^j  = \left( I-\tau A \right)^j \left( U-U^0 \right)$ so $\norm{U-U^j} \leq \textcolor{blue}{\norm{I-\tau A}}^j\norm{U-U^0}$. $\textcolor{blue}{\norm{I-\tau A}} = \sigma_1 = \abs{\lambda_1}$ as symmetric. If $\lambda \in [\alpha, \beta]$ then $\lambda_1 \leq \max{ \left\{ \abs{1-\tau \alpha},\abs{1-\tau \beta} \right\} }$. Attained when $\tau = 2/(\alpha + \beta) \rightarrow \lambda_1  = \frac{\beta - \alpha}{ \beta + \alpha} $. For $-u''+cu=f$ we have $\lambda_k = c+ \frac{4}{h^2} \sin^2\left( \frac{k \pi h}{2}  \right)$. Lower bound via noting $\sin(y) \geq \frac{2 \sqrt 2}{\pi} y$ at $y= \frac{\pi}{4} \rightarrow \lambda_k \geq c+ 8$
\textbf{Errors:} $(AV,V)_h \geq ||D_x^{-}V]|^2_h$ \& PF Ineq $\rightarrow (AV,V)_h \geq \norm{V}_h^2/c_\star$. Then $(AV,V)_h\left( 1+c_\star \right) \geq \norm{V}_{1,h}^2 \rightarrow (AV,V)_h \geq c_0 \norm{V}_{1,h}^2$. Now $c_0 \norm{V}_{1,h}^2 \leq (AV,V)_h \leq \norm{f}_h \norm{V}_h \leq \norm{f}_h \norm{V}_{1,h} \rightarrow \textcolor{blue}{\norm{V}_{1,h} \leq \norm{f}_h/c_0}$. (Use $f:= AV \rightarrow \textcolor{blue}{ \norm{e}_{1,h} \leq \norm{T}_h /c_0}$)
\end{document}
